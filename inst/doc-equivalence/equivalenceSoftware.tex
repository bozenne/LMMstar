% Created 2025-10-21 ti 12:07
% Intended LaTeX compiler: pdflatex
\documentclass[12pt]{article}

%%%% settings when exporting code %%%% 

\usepackage{listings}
\lstdefinestyle{code-small}{
backgroundcolor=\color{white}, % background color for the code block
basicstyle=\ttfamily\small, % font used to display the code
commentstyle=\color[rgb]{0.5,0,0.5}, % color used to display comments in the code
keywordstyle=\color{black}, % color used to highlight certain words in the code
numberstyle=\ttfamily\tiny\color{gray}, % color used to display the line numbers
rulecolor=\color{black}, % color of the frame
stringstyle=\color[rgb]{0,.5,0},  % color used to display strings in the code
breakatwhitespace=false, % sets if automatic breaks should only happen at whitespace
breaklines=true, % sets automatic line breaking
columns=fullflexible,
frame=single, % adds a frame around the code (non,leftline,topline,bottomline,lines,single,shadowbox)
keepspaces=true, % % keeps spaces in text, useful for keeping indentation of code
literate={~}{$\sim$}{1}, % symbol properly display via latex
numbers=none, % where to put the line-numbers; possible values are (none, left, right)
numbersep=10pt, % how far the line-numbers are from the code
showspaces=false,
showstringspaces=false,
stepnumber=1, % the step between two line-numbers. If it's 1, each line will be numbered
tabsize=1,
xleftmargin=0cm,
emph={anova,apply,class,coef,colnames,colNames,colSums,dim,dcast,for,ggplot,head,if,ifelse,is.na,lapply,list.files,library,logLik,melt,plot,require,rowSums,sapply,setcolorder,setkey,str,summary,tapply},
aboveskip = \medskipamount, % define the space above displayed listings.
belowskip = \medskipamount, % define the space above displayed listings.
lineskip = 0pt} % specifies additional space between lines in listings
\lstset{style=code-small}
%%%% packages %%%%%

\usepackage[utf8]{inputenc}
\usepackage[T1]{fontenc}
\usepackage{lmodern}
\usepackage{textcomp}
\usepackage{color}
\usepackage{graphicx}
\usepackage{grffile}
\usepackage{wrapfig}
\usepackage{rotating}
\usepackage{longtable}
\usepackage{multirow}
\usepackage{multicol}
\usepackage{changes}
\usepackage{pdflscape}
\usepackage{geometry}
\usepackage[normalem]{ulem}
\usepackage{amssymb}
\usepackage{amsmath}
\usepackage{amsfonts}
\usepackage{dsfont}
\usepackage{array}
\usepackage{ifthen}
\usepackage{hyperref}
\usepackage{natbib}
\RequirePackage{setspace} % to modify the space between lines - incompatible with footnote in beamer
\renewcommand{\baselinestretch}{1.1}
\geometry{a4paper, left=10mm, right=10mm, top=10mm}
\usepackage{titlesec}
\usepackage{etoolbox}

\makeatletter
\patchcmd{\ttlh@hang}{\parindent\z@}{\parindent\z@\leavevmode}{}{}
\patchcmd{\ttlh@hang}{\noindent}{}{}{}
\makeatother
\RequirePackage{colortbl} % arrayrulecolor to mix colors
\definecolor{myorange}{rgb}{1,0.2,0}
\definecolor{mypurple}{rgb}{0.7,0,8}
\definecolor{mycyan}{rgb}{0,0.6,0.6}
\newcommand{\lightblue}{blue!50!white}
\newcommand{\darkblue}{blue!80!black}
\newcommand{\darkgreen}{green!50!black}
\newcommand{\darkred}{red!50!black}
\definecolor{gray}{gray}{0.5}
\hypersetup{
citecolor=[rgb]{0,0.5,0},
urlcolor=[rgb]{0,0,0.5},
linkcolor=[rgb]{0,0,0.5},
}
\newenvironment{note}{\small \color{gray}\fontfamily{lmtt}\selectfont}{\par}
\newenvironment{activity}{\color{orange}\fontfamily{qzc}\selectfont}{\par}
\RequirePackage{pifont}
\RequirePackage{relsize}
\newcommand{\Cross}{{\raisebox{-0.5ex}%
{\relsize{1.5}\ding{56}}}\hspace{1pt} }
\newcommand{\Valid}{{\raisebox{-0.5ex}%
{\relsize{1.5}\ding{52}}}\hspace{1pt} }
\newcommand{\CrossR}{ \textcolor{red}{\Cross} }
\newcommand{\ValidV}{ \textcolor{green}{\Valid} }
\usepackage{stackengine}
\usepackage{scalerel}
\newcommand\Warning[1][3ex]{%
\renewcommand\stacktype{L}%
\scaleto{\stackon[1.3pt]{\color{red}$\triangle$}{\tiny\bfseries !}}{#1}%
\xspace
}
\RequirePackage{fancyvrb}
\DefineVerbatimEnvironment{verbatim}{Verbatim}{fontsize=\small,formatcom = {\color[rgb]{0.5,0,0}}}
\definecolor{grayR}{HTML}{8A8990}
\definecolor{grayL}{HTML}{C4C7C9}
\definecolor{blueM}{HTML}{1F63B5}
\newcommand{\Rlogo}[1][0.07]{
\begin{tikzpicture}[scale=#1]
\shade [right color=grayR,left color=grayL,shading angle=60]
(-3.55,0.3) .. controls (-3.55,1.75)
and (-1.9,2.7) .. (0,2.7) .. controls (2.05,2.7)
and (3.5,1.6) .. (3.5,0.3) .. controls (3.5,-1.2)
and (1.55,-2) .. (0,-2) .. controls (-2.3,-2)
and (-3.55,-0.75) .. cycle;

\fill[white]
(-2.15,0.2) .. controls (-2.15,1.2)
and (-0.7,1.8) .. (0.5,1.8) .. controls (2.2,1.8)
and (3.1,1.2) .. (3.1,0.2) .. controls (3.1,-0.75)
and (2.4,-1.45) .. (0.5,-1.45) .. controls (-1.1,-1.45)
and (-2.15,-0.7) .. cycle;

\fill[blueM]
(1.75,1.25) -- (-0.65,1.25) -- (-0.65,-2.75) -- (0.55,-2.75) -- (0.55,-1.15) --
(0.95,-1.15)  .. controls (1.15,-1.15)
and (1.5,-1.9) .. (1.9,-2.75) -- (3.25,-2.75)  .. controls (2.2,-1)
and (2.5,-1.2) .. (1.8,-0.95) .. controls (2.6,-0.9)
and (2.85,-0.35) .. (2.85,0.2) .. controls (2.85,0.7)
and (2.5,1.2) .. cycle;

\fill[white]  (1.4,0.4) -- (0.55,0.4) -- (0.55,-0.3) -- (1.4,-0.3).. controls (1.75,-0.3)
and (1.75,0.4) .. cycle;

\end{tikzpicture}
}
\RequirePackage{epstopdf} % to be able to convert .eps to .pdf image files
\RequirePackage{capt-of} %
\RequirePackage{caption} % newlines in graphics
\RequirePackage{tikz-cd} % graph
\RequirePackage{booktabs} % for nice lines in table (e.g. toprule, bottomrule, midrule, cmidrule)
\RequirePackage{amsmath}
\RequirePackage{algorithm}
\RequirePackage[noend]{algpseudocode}
\RequirePackage{dsfont}
\RequirePackage{amsmath,stmaryrd,graphicx}
\RequirePackage{prodint} % product integral symbol (\PRODI)
\usepackage{ifthen}
\usepackage{xifthen}
\usepackage{xargs}
\usepackage{xspace}
\newcommand\defOperator[7]{%
\ifthenelse{\isempty{#2}}{
\ifthenelse{\isempty{#1}}{#7{#3}#4}{#7{#3}#4 \left#5 #1 \right#6}
}{
\ifthenelse{\isempty{#1}}{#7{#3}#4_{#2}}{#7{#3}#4_{#1}\left#5 #2 \right#6}
}
}
\newcommand\defUOperator[5]{%
\ifthenelse{\isempty{#1}}{
#5\left#3 #2 \right#4
}{
\ifthenelse{\isempty{#2}}{\underset{#1}{\operatornamewithlimits{#5}}}{
\underset{#1}{\operatornamewithlimits{#5}}\left#3 #2 \right#4}
}
}
\newcommand{\defBoldVar}[2]{
\ifthenelse{\equal{#2}{T}}{\boldsymbol{#1}}{\mathbf{#1}}
}
\newcommandx\Esp[2][1=,2=]{\defOperator{#1}{#2}{E}{}{\lbrack}{\rbrack}{\mathbb}}
\newcommandx\Prob[2][1=,2=]{\defOperator{#1}{#2}{P}{}{\lbrack}{\rbrack}{\mathbb}}
\newcommandx\Qrob[2][1=,2=]{\defOperator{#1}{#2}{Q}{}{\lbrack}{\rbrack}{\mathbb}}
\newcommandx\Var[2][1=,2=]{\defOperator{#1}{#2}{V}{ar}{\lbrack}{\rbrack}{\mathbb}}
\newcommandx\Cov[2][1=,2=]{\defOperator{#1}{#2}{C}{ov}{\lbrack}{\rbrack}{\mathbb}}
\newcommandx\Binom[2][1=,2=]{\defOperator{#1}{#2}{B}{}{(}{)}{\mathcal}}
\newcommandx\Gaus[2][1=,2=]{\defOperator{#1}{#2}{N}{}{(}{)}{\mathcal}}
\newcommandx\Wishart[2][1=,2=]{\defOperator{#1}{#2}{W}{ishart}{(}{)}{\mathcal}}
\newcommandx\Likelihood[2][1=,2=]{\defOperator{#1}{#2}{L}{}{(}{)}{\mathcal}}
\newcommandx\logLikelihood[2][1=,2=]{\defOperator{#1}{#2}{\ell}{}{(}{)}{}}
\newcommandx\Information[2][1=,2=]{\defOperator{#1}{#2}{I}{}{(}{)}{\mathcal}}
\newcommandx\Hessian[2][1=,2=]{\defOperator{#1}{#2}{H}{}{(}{)}{\mathcal}}
\newcommandx\Score[2][1=,2=]{\defOperator{#1}{#2}{S}{}{(}{)}{\mathcal}}
\newcommandx\Vois[2][1=,2=]{\defOperator{#1}{#2}{V}{}{(}{)}{\mathcal}}
\newcommandx\IF[2][1=,2=]{\defOperator{#1}{#2}{IF}{}{(}{)}{\mathcal}}
\newcommandx\Ind[1][1=]{\defOperator{}{#1}{1}{}{(}{)}{\mathds}}
\newcommandx\Max[2][1=,2=]{\defUOperator{#1}{#2}{(}{)}{min}}
\newcommandx\Min[2][1=,2=]{\defUOperator{#1}{#2}{(}{)}{max}}
\newcommandx\argMax[2][1=,2=]{\defUOperator{#1}{#2}{(}{)}{argmax}}
\newcommandx\argMin[2][1=,2=]{\defUOperator{#1}{#2}{(}{)}{argmin}}
\newcommandx\cvD[2][1=D,2=n \rightarrow \infty]{\xrightarrow[#2]{#1}}
\newcommandx\Hypothesis[2][1=,2=]{
\ifthenelse{\isempty{#1}}{
\mathcal{H}
}{
\ifthenelse{\isempty{#2}}{
\mathcal{H}_{#1}
}{
\mathcal{H}^{(#2)}_{#1}
}
}
}
\newcommandx\dpartial[4][1=,2=,3=,4=\partial]{
\ifthenelse{\isempty{#3}}{
\frac{#4 #1}{#4 #2}
}{
\left.\frac{#4 #1}{#4 #2}\right\rvert_{#3}
}
}
\newcommandx\dTpartial[3][1=,2=,3=]{\dpartial[#1][#2][#3][d]}
\newcommandx\ddpartial[3][1=,2=,3=]{
\ifthenelse{\isempty{#3}}{
\frac{\partial^{2} #1}{\partial #2^2}
}{
\frac{\partial^2 #1}{\partial #2\partial #3}
}
}
\newcommand\Real{\mathbb{R}}
\newcommand\Rational{\mathbb{Q}}
\newcommand\Natural{\mathbb{N}}
\newcommand\trans[1]{{#1}^\intercal}%\newcommand\trans[1]{{\vphantom{#1}}^\top{#1}}
\newcommand{\independent}{\mathrel{\text{\scalebox{1.5}{$\perp\mkern-10mu\perp$}}}}
\newcommand\half{\frac{1}{2}}
\newcommand\normMax[1]{\left|\left|#1\right|\right|_{max}}
\newcommand\normTwo[1]{\left|\left|#1\right|\right|_{2}}
\newcommand\Veta{\boldsymbol{\eta}}
\newcommand{\Model}{\mathcal{M}}
\newcommand{\ModelHat}{\widehat{\mathcal{M}}}
\newcommand{\param}{\Theta}
\newcommand{\paramHat}{\widehat{\param}}
\newcommand{\paramCon}{\widetilde{\param}}
\newcommand{\Vparam}{\boldsymbol{\param}}
\newcommand{\VparamT}{\Vparam_0}
\newcommand{\VparamHat}{\boldsymbol{\paramHat}}
\newcommand{\VparamCon}{\boldsymbol{\paramCon}}
\newcommand{\X}{X}
\newcommand{\x}{x}
\newcommand{\VX}{\boldsymbol{X}}
\newcommand{\Vx}{\boldsymbol{x}}
\newcommand{\Y}{Y}
\newcommand{\y}{y}
\newcommand{\VY}{\boldsymbol{Y}}
\newcommand{\Vy}{\boldsymbol{y}}
\newcommand{\Vvarepsilon}{\boldsymbol{\varepsilon}}
\newcommand{\Z}{Z}
\newcommand{\z}{z}
\newcommand{\VZ}{\boldsymbol{Z}}
\newcommand{\Vz}{\boldsymbol{z}}
\author{Brice Ozenne}
\date{\today}
\title{Comparison with other \Rlogo packages}
\hypersetup{
 colorlinks=true,
 pdfauthor={Brice Ozenne},
 pdftitle={Comparison with other \Rlogo packages},
 pdfkeywords={},
 pdfsubject={},
 pdfcreator={Emacs 30.1 (Org mode 9.7.11)},
 pdflang={English}
 }
\begin{document}

\maketitle
This vignette make connexions between the output of a linear mixed
model and well-known tests (t.test, ANCOVA, Pearson's correlation,
Bartlett's test, \ldots). The second part of the vignette make
connexion with other \Rlogo packages.
\section{nlme package}
\label{sec:orgf1cec13}

The model class obtained with the \texttt{lmm} function overlaps the model
class of the \texttt{lme} and \texttt{gls} functions from the nlme package.
\begin{lstlisting}[language=r,numbers=none]
library(nlme)
\end{lstlisting}

For instance, the compound symmetry is equivalent to \texttt{corCompSymm}
correlation structure, or to a random intercept model (when the within
subject correlation is positive):
\begin{lstlisting}[language=r,numbers=none]
eRI.lmm <- lmm(bmd ~ visit*grp, structure = "RE",
               data = calciumL, repetition = ~visit|girl)
eCS.gls <- gls(bmd ~ visit*grp, correlation = corCompSymm(form=~visit|girl),
               data = calciumL, na.action = na.omit)
eCS.lme <- lme(bmd ~ visit*grp, random = ~1|girl,
               data = calciumL, na.action = na.omit)
logLik(eRI.lmm)
logLik(eCS.lme)
logLik(eCS.gls)
\end{lstlisting}

\phantomsection
\label{}
\begin{verbatim}
[1] -2297.3
'log Lik.' -2297.3 (df=12)
'log Lik.' -2297.3 (df=12)
\end{verbatim}


The estimated random effect also match:
\begin{lstlisting}[language=r,numbers=none]
range(ranef(eRI.lmm)-ranef(eCS.lme))
\end{lstlisting}

\phantomsection
\label{}
\begin{verbatim}
[1] -6.6939e-08  3.1497e-08
\end{verbatim}


Unstructured residual covariance matrix can also be obtained with
\texttt{gls}:
\begin{lstlisting}[language=r,numbers=none]
eUN.gls <- gls(bmd ~ visit*grp,
               correlation = corSymm(form=~as.numeric(visit)|girl),
               weights = varIdent(form=~1|visit),
               data = calciumL, na.action = na.omit)
eUN.lmm <- lmm(bmd ~ visit*grp, structure = "UN",
               data = calciumL, repetition = ~visit|girl)
logLik(eUN.gls)
logLik(eUN.lmm)
\end{lstlisting}

\phantomsection
\label{}
\begin{verbatim}
'log Lik.' -2218.5 (df=25)
[1] -2218.5
\end{verbatim}


\clearpage
\subsection{lme4 package}
\label{sec:org4a480f1}

The model class obtained with the \texttt{lmm} function overlaps the model
class of the \texttt{lmer} function from the lme4 package.
\begin{lstlisting}[language=r,numbers=none]
library(lme4)
library(lmerTest)
\end{lstlisting}

For instance, the compound symmetry is equivalent to a random
intercept model (when the within subject correlation is positive):
\begin{lstlisting}[language=r,numbers=none]
eRI.lmer <- lmer(bmd ~ visit*grp + (1|girl), data = calciumL)
logLik(eRI.lmer)
logLik(eRI.lmm)
\end{lstlisting}

\phantomsection
\label{}
\begin{verbatim}
'log Lik.' -2297.3 (df=12)
[1] -2297.3
\end{verbatim}


The estimated random effects match:
\begin{lstlisting}[language=r,numbers=none]
range(ranef(eRI.lmm)-ranef(eRI.lmer)$girl)
\end{lstlisting}

\phantomsection
\label{}
\begin{verbatim}
[1] -7.3817e-08  3.4754e-08
\end{verbatim}


Nested random effects correspond to block unstructured:
\begin{lstlisting}[language=r,numbers=none]
calciumLB <- transform(calciumL, baseline = visit==1)

eNRI.lmm <- lmm(bmd ~ visit*grp, structure = RE(~(1|girl/baseline)),
                repetition = ~visit|girl,
                data = calciumLB)
eNRI.lmer <- lmer(bmd ~ visit*grp + (1|girl/baseline),
                  data = calciumLB)
logLik(eNRI.lmer)
logLik(eNRI.lmm)
\end{lstlisting}

\phantomsection
\label{}
\begin{verbatim}
'log Lik.' -2282.1 (df=13)
[1] -2282.1
\end{verbatim}


And the estimated random effects still match:
\begin{lstlisting}[language=r,numbers=none]
eRanefNRI.lmm <- ranef(eNRI.lmm, format = "wide")
eRanefNRI.lmer <- ranef(eNRI.lmer)
## girl
range(eRanefNRI.lmm$estimate-eRanefNRI.lmer$girl)
## baseline
eRanefNRI2.lmm <- c(eRanefNRI.lmm$estimate.FALSE,eRanefNRI.lmm$estimate.TRUE)
eRanefNRI2.lmer <- ranef(eNRI.lmer)$`baseline:girl`
range(na.omit(eRanefNRI2.lmm)-eRanefNRI2.lmer)
\end{lstlisting}

\phantomsection
\label{}
\begin{verbatim}
[1] -1.2487e-06  1.5182e-06
[1] -2.4547e-06  1.9463e-06
\end{verbatim}


\clearpage

An unstructure residual covariance matrix can also be obtained using
random slopes:
\begin{lstlisting}[language=r,numbers=none]
eUN.lmer <- lmer(bmd ~ visit*grp + (0 + visit|girl),
                 data = calciumL,
                 control = lmerControl(check.nobs.vs.nRE = "ignore"))
logLik(eUN.lmer)
logLik(eUN.lmm)
\end{lstlisting}

\phantomsection
\label{}
\begin{verbatim}
Advarselsbesked:
I checkConv(attr(opt, "derivs"), opt$par, ctrl = control$checkConv,  :
  Model failed to converge with max|grad| = 0.0101162 (tol = 0.002, component 1)
'log Lik.' -2218.5 (df=26)
[1] -2218.5
\end{verbatim}


The uncertainty is quantified in a slightly different way, e.g.:
\begin{lstlisting}[language=r,numbers=none]
anova(eUN.lmm)
\end{lstlisting}

\phantomsection
\label{}
\begin{verbatim}
		Wald F-tests 

                statistic        df p.value    
mean: visit       111.043 (4, 96.0)  <1e-04 ***
      grp           0.764 (1,109.9)  0.3840    
      visit:grp     2.791 (4, 96.5)  0.0305   *
\end{verbatim}


is very similar but not identical to:
\begin{lstlisting}[language=r,numbers=none]
anova(eUN.lmer) ## only the last line is comparable
\end{lstlisting}

\phantomsection
\label{}
\begin{verbatim}
Type III Analysis of Variance Table with Satterthwaite's method
          Sum Sq Mean Sq NumDF DenDF F value Pr(>F)    
visit      65527   16382     4  96.8  258.07 <2e-16 ***
grp          162     162     1 109.3    2.55   0.11    
visit:grp    710     177     4  96.8    2.79   0.03 *  
---
Signif. codes:  0 '***' 0.001 '**' 0.01 '*' 0.05 '.' 0.1 ' ' 1
\end{verbatim}


It is also possible to fit cross-random effects such as:
\begin{lstlisting}[language=r,numbers=none]
data("Penicillin")
eCRI.lmer <- lmer(diameter ~ 1 + (1|plate) + (1|sample), Penicillin)
logLik(eCRI.lmer)
\end{lstlisting}

\phantomsection
\label{}
\begin{verbatim}
'log Lik.' -165.43 (df=4)
\end{verbatim}



using \texttt{lmm}:
\begin{lstlisting}[language=r,numbers=none]
Penicillin$index <- paste(Penicillin$sample,Penicillin$plate,sep=".")
Penicillin$id <- 1

eCRI.lmm <- lmm(diameter ~ 1 + (1|plate) + (1|sample), data = Penicillin)
logLik(eCRI.lmm)
\end{lstlisting}

\phantomsection
\label{}
\begin{verbatim}
[1] -165.43
\end{verbatim}


Despite \texttt{lmm} being significantly slower, the loglikelihood and random
effect still match:
\begin{lstlisting}[language=r,numbers=none]
range(ranef(eCRI.lmm)$estimate-rbind(ranef(eCRI.lmer)$plate,ranef(eCRI.lmer)$sample))
\end{lstlisting}

\phantomsection
\label{}
\begin{verbatim}
[1] -4.4050e-07  6.0499e-07
\end{verbatim}


\clearpage
\subsection{mmrm package}
\label{sec:orged18e50}

The package \texttt{mmrm} is an alternative implementation of mixed models
specified via covariance structures:
\begin{lstlisting}[language=r,numbers=none]
library(mmrm)
e.mmrm <- mmrm(
  formula = FEV1 ~ RACE + SEX + ARMCD * AVISIT + us(AVISIT | USUBJID),
  data = fev_data
)
\end{lstlisting}

\phantomsection
\label{}
\begin{verbatim}
mmrm() registered as car::Anova extension
\end{verbatim}


It leads nearly identical results compared to \texttt{lmm}:
\begin{lstlisting}[language=r,numbers=none]
e.lmm <- lmm(
  FEV1 ~ RACE + SEX + ARMCD * AVISIT,
  repetition = ~ AVISIT | USUBJID, structure = "UN",
  data = fev_data, type.information = "expected"
)
\end{lstlisting}
\begin{lstlisting}[language=r,numbers=none]
logLik(e.mmrm) - logLik(e.lmm)
range(coef(e.mmrm) - coef(e.lmm))
range(vcov(e.mmrm) - vcov(e.lmm))
\end{lstlisting}

\phantomsection
\label{}
\begin{verbatim}
[1] -2.5413e-06
[1] -0.00018345  0.00016319
[1] -0.00039810  0.00020542
\end{verbatim}


The main differences are:
\begin{itemize}
\item \texttt{mmrm} uses the expected information matrix to quantify uncertainty
instead of the observed information matrix.
\item \texttt{mmrm} implements the Kenward and Roger method for computing the degrees of
freedom and not only the Satterthwaite approximation
\item \texttt{mmrm} implements different covariance patterns
\item \texttt{mmrm} is faster and probably more memorry efficient
\item \texttt{mmrm} has currently fewer post-processing methods (e.g. adjustment
multiple comparisons when testing several model parameters).
\end{itemize}

\clearpage
\subsection{emmeans package}
\label{sec:org2b0f27a}

To illustrate a key difference between the emmeans package and the
\texttt{effects.lmm} function we consider an informative and unbalanced group
variable:
\begin{lstlisting}[language=r,numbers=none]
gastricbypassLB$group2 <- gastricbypassLB$weight1>150
\end{lstlisting}

Since \texttt{lmm}:
\begin{lstlisting}[language=r,numbers=none]
eCS.lmm_2 <- lmm(glucagonAUC ~ visit*group2, repetition =~visit|id, structure = "CS", data = gastricbypassLB)
logLik(eCS.lmm_2)
\end{lstlisting}

\phantomsection
\label{}
\begin{verbatim}
[1] -315.2
\end{verbatim}


we will use the equivalent with the random effect specification:

\begin{lstlisting}[language=r,numbers=none]
eRI.lmer_2 <- lmer(glucagonAUC ~ visit*group2 + (1|id), data = gastricbypassLB)
logLik(eRI.lmer_2)
\end{lstlisting}

\phantomsection
\label{}
\begin{verbatim}
'log Lik.' -315.2 (df=10)
\end{verbatim}


While the two models are equivalent, the average outcome output by
\texttt{effects}:
\begin{lstlisting}[language=r,numbers=none]
effects(eCS.lmm_2, variable = NULL)
\end{lstlisting}

\phantomsection
\label{}
\begin{verbatim}
		Average counterfactual outcome

      estimate    se   df  lower  upper
(t=1)   32.317 4.426 64.3 23.476 41.158
(t=2)   29.653 4.535 65.2 20.598 38.709
(t=3)   77.308 4.535 65.1  68.25 86.366
(t=4)    51.95 4.426 64.3 43.109 60.791
\end{verbatim}


substantially differ from the one of emmeans:
\begin{lstlisting}[language=r,numbers=none]
library(emmeans)
emmeans(eRI.lmer_2, specs=~visit)
\end{lstlisting}

\phantomsection
\label{}
\begin{verbatim}
NOTE: Results may be misleading due to involvement in interactions
 visit emmean   SE   df lower.CL upper.CL
 1       33.6 5.53 64.2     22.6     44.7
 2       32.0 5.57 64.4     20.9     43.2
 3       70.0 5.57 64.4     58.9     81.1
 4       47.2 5.53 64.2     36.1     58.2

Results are averaged over the levels of: group2 
Degrees-of-freedom method: kenward-roger 
Confidence level used: 0.95
\end{verbatim}

This is because when averaging over the level of a covariate, emmeans
considers \emph{balanced groups}. In the example, the groups are not
balanced:
\begin{lstlisting}[language=r,numbers=none]
table(gastricbypassLB$group2)/NROW(gastricbypassLB)
\end{lstlisting}

\phantomsection
\label{}
\begin{verbatim}

FALSE  TRUE 
  0.8   0.2
\end{verbatim}


Based on the group and timepoint specific means:
\begin{lstlisting}[language=r,numbers=none]
eCS.elmm_2 <- model.tables(effects(eCS.lmm_2, variable = "group2"))
eCS.elmm_2
\end{lstlisting}

\phantomsection
\label{}
\begin{verbatim}
  group2 visit estimate     se     df  lower  upper    p.value
1  FALSE     1   31.430 4.9484 64.349 21.545 41.314 2.4688e-08
2  FALSE     2   28.067 5.0996 65.383 17.884 38.251 6.6737e-07
3  FALSE     3   82.173 5.1008 65.211 71.986 92.359 0.0000e+00
4  FALSE     4   55.126 4.9484 64.349 45.241 65.010 0.0000e+00
5   TRUE     1   35.864 9.8967 64.349 16.095 55.633 5.7374e-04
6   TRUE     2   35.997 9.8967 64.349 16.228 55.766 5.4953e-04
7   TRUE     3   57.848 9.8967 64.349 38.079 77.617 1.8339e-07
8   TRUE     4   39.246 9.8967 64.349 19.477 59.015 1.8651e-04
\end{verbatim}


We illustrate the difference:
\begin{itemize}
\item emmeans:
\end{itemize}
\begin{lstlisting}[language=r,numbers=none]
0.5*eCS.elmm_2[eCS.elmm_2$group2==FALSE,"estimate"]+0.5*eCS.elmm_2[eCS.elmm_2$group2==TRUE,"estimate"]
\end{lstlisting}

\phantomsection
\label{}
\begin{verbatim}
[1] 33.647 32.032 70.010 47.186
\end{verbatim}


\begin{itemize}
\item effects:
\end{itemize}
\begin{lstlisting}[language=r,numbers=none]
0.8*eCS.elmm_2[eCS.elmm_2$group2==FALSE,"estimate"]+0.2*eCS.elmm_2[eCS.elmm_2$group2==TRUE,"estimate"]
\end{lstlisting}

\phantomsection
\label{}
\begin{verbatim}
[1] 32.317 29.653 77.308 51.950
\end{verbatim}


The "emmeans" approach gives equal "weight" to the expected value of
both group:
\begin{lstlisting}[language=r,numbers=none]
mu.group1 <-  as.double(coef(e.group)["(Intercept)"])
mu.group2 <-  as.double(coef(e.group)["(Intercept)"] + coef(e.group)["group2TRUE"])
p.group1 <- 14/20          ; p.group2 <- 6/20
c(emmeans = (mu.group1+mu.group2)/2, predict = mu.group1 * p.group1 + mu.group2 * p.group2)
\end{lstlisting}

\phantomsection
\label{}
\begin{verbatim}
 emmeans  predict 
4.450435 4.514352
\end{verbatim}


\clearpage
\subsection{effectsize package (\(R^2\) or \(\eta^2\))}
\label{sec:org1476e5a}

Partial \(\eta^2\) can be computed based on \texttt{lmer} using the effectsize package:
\begin{lstlisting}[language=r,numbers=none]
library(effectsize)
eta_squared(eCS.lmer)
cat("\n")
\end{lstlisting}

\phantomsection
\label{}
\begin{verbatim}
# Effect Size for ANOVA (Type III)

Parameter   | Eta2 (partial) |       95% CI
-------------------------------------------
visit       |           0.64 | [0.50, 1.00]
group       |           0.01 | [0.00, 1.00]
visit:group |           0.19 | [0.03, 1.00]

- One-sided CIs: upper bound fixed at
\end{verbatim}


and are approximately equal to what one can compute "manually":
\begin{lstlisting}[language=r,numbers=none]
eCS.Wald <- anova(eCS.lmm)$multivariate
eCS.Wald$df.num*eCS.Wald$statistic/(eCS.Wald$df.num*eCS.Wald$statistic+eCS.Wald$df.denom)
\end{lstlisting}

\phantomsection
\label{}
\begin{verbatim}
[1] 0.335374 0.033811 0.186290
\end{verbatim}


The will not be true for heteroschedastic models:
\begin{lstlisting}[language=r,numbers=none]
eUN.Wald <- anova(eUN.lmm)$multivariate
eUN.Wald$df.num*eUN.Wald$statistic/(eUN.Wald$df.num*eUN.Wald$statistic+eUN.Wald$df.denom)
\end{lstlisting}

\phantomsection
\label{}
\begin{verbatim}
[1] 0.50787 0.17905 0.32380
\end{verbatim}


compared to:
\begin{lstlisting}[language=r,numbers=none]
eta_squared(eUN.lmer)
cat("\n")
\end{lstlisting}

\phantomsection
\label{}
\begin{verbatim}
# Effect Size for ANOVA (Type III)

Parameter   | Eta2 (partial) |       95% CI
-------------------------------------------
visit       |           0.76 | [0.54, 1.00]
group       |           0.01 | [0.00, 1.00]
visit:group |           0.32 | [0.00, 1.00]

- One-sided CIs: upper bound fixed at
\end{verbatim}


But in that case both may be misleading as the proportion of explained
variance is not clearly defined.
\subsection{MuMIn package (\(R^2\))}
\label{sec:org29c722b}

\begin{lstlisting}[language=r,numbers=none]
library(MuMIn)
r.squaredGLMM(eCS.lmer)
cat("\n")
\end{lstlisting}

\phantomsection
\label{}
\begin{verbatim}
         R2m     R2c
[1,] 0.51728 0.62222
\end{verbatim}


To reproduce these R2, we extract from the random intercept model:
\begin{itemize}
\item the residual variance
\end{itemize}
\begin{lstlisting}[language=r,numbers=none]
sigmaW <- sigma(eCS.lmm)[1,1]-sigma(eCS.lmm)[1,2]
\end{lstlisting}

\begin{itemize}
\item the variance of the random effect
\end{itemize}
\begin{lstlisting}[language=r,numbers=none]
sigmaB <- sigma(eCS.lmm)[1,2]
\end{lstlisting}

\begin{itemize}
\item the variance of the fitted values:
\end{itemize}
\begin{lstlisting}[language=r,numbers=none]
sigma2_XB <- var(fitted(eCS.lmm))
\end{lstlisting}

and evalutae the ratios:
\begin{lstlisting}[language=r,numbers=none]
c(R2m = sigma2_XB/(sigmaW + sigmaB + sigma2_XB),
  R2c = (sigma2_XB + sigmaB)/(sigmaW + sigmaB + sigma2_XB))
\end{lstlisting}

\phantomsection
\label{}
\begin{verbatim}
    R2m     R2c 
0.52549 0.62865
\end{verbatim}
\subsection{stats package (partial residuals)}
\label{sec:orgb3dd408}

The function \texttt{residuals.lm} can be used to extract partial residuals
from \texttt{lm} objects. For instance:
\begin{lstlisting}[language=r,numbers=none]
gastricbypassW$group <- as.factor(as.numeric(gastricbypassW$id)%%2)
eIID.lm <- lm(weight4 ~ group + weight1, data = gastricbypassW)
pRes.lm <- residuals(eIID.lm, type = "partial")
head(pRes.lm)
\end{lstlisting}

\phantomsection
\label{}
\begin{verbatim}
      group  weight1
1   7.19282   3.6648
2  -0.20504  31.7052
3   0.60631 -17.3352
4   6.44389  22.7052
5  -1.59403 -16.7352
6 -18.23382   8.4052
\end{verbatim}


Those generally differ (by a constant) from the one provided by
\texttt{residuals.lmm}:
\begin{lstlisting}[language=r,numbers=none]
eIID.lmm <- lmm(weight4 ~ group + weight1, data = gastricbypassW)
(residuals(eIID.lmm, type = "partial", variable = "group") - pRes.lm[,"group"])
(residuals(eIID.lmm, type = "partial", variable = "weight1") - pRes.lm[,"weight1"])
\end{lstlisting}

\phantomsection
\label{}
\begin{verbatim}
     1      2      3      4      5      6      7      8      9     10     11     12     13     14 
2.0702 2.0702 2.0702 2.0702 2.0702 2.0702 2.0702 2.0702 2.0702 2.0702 2.0702 2.0702 2.0702 2.0702 
    15     16     17     18     19     20 
2.0702 2.0702 2.0702 2.0702 2.0702 2.0702
     1      2      3      4      5      6      7      8      9     10     11     12     13     14 
106.22 106.22 106.22 106.22 106.22 106.22 106.22 106.22 106.22 106.22 106.22 106.22 106.22 106.22 
    15     16     17     18     19     20 
106.22 106.22 106.22 106.22 106.22 106.22
\end{verbatim}


Indeed, \texttt{residuals.lm} centers the design matrix of the variable
relative to which the partial residuals are computed:
\begin{lstlisting}[language=r,numbers=none]
coef(eIID.lm)["group1"] * mean(gastricbypassW$group=="1")
coef(eIID.lm)["weight1"] * mean(gastricbypassW$weight1)
\end{lstlisting}

\phantomsection
\label{}
\begin{verbatim}
group1 
2.0702
weight1 
 106.22
\end{verbatim}


For continuous variable with a linear effect, these residuals can be
obtained by setting the \texttt{type} argument to \texttt{"partial-center"}:
\begin{lstlisting}[language=r,numbers=none]
(residuals(eIID.lmm, type = "partial-center", variable = "weight1") - pRes.lm[,"weight1"])
\end{lstlisting}

\phantomsection
\label{}
\begin{verbatim}
          1           2           3           4           5           6           7           8 
 1.7675e-13  6.7502e-14 -6.3949e-14  5.6843e-14 -3.9080e-14  8.1712e-14 -3.7303e-14  5.9508e-14 
          9          10          11          12          13          14          15          16 
-4.2633e-14  4.4409e-14 -2.9310e-14  5.5123e-14 -4.6185e-14  4.4409e-14 -4.2633e-14  4.6185e-14 
         17          18          19          20 
-3.9968e-14  5.3291e-14 -1.4211e-14  3.5527e-14
\end{verbatim}


\Warning When evaluating the partial residuals relative to categorical
variables, interactions, or non-linear terms, the output obtained with
\texttt{partial-center} will not match the one of \texttt{residuals.lm}. Indeed
\texttt{partial-center} will, when numeric, center the original variable
whereas \texttt{residuals.lm} will center the column relative to the
coefficient in the design matrix.
\section*{References}
\label{sec:org66b89bf}
\begingroup
\renewcommand{\section}[2]{}

\bibliographystyle{apalike}
\bibliography{bibliography}

\endgroup

\clearpage

\appendix
\titleformat{\section}
{\normalfont\Large\bfseries}{Appendix~\thesection}{1em}{}

\renewcommand{\thefigure}{\Alph{figure}}
\renewcommand{\thetable}{\Alph{table}}
\renewcommand{\theequation}{\Alph{equation}}

\setcounter{figure}{0}    
\setcounter{table}{0}    
\setcounter{equation}{0}    
\end{document}
